\documentclass[10pt]{article}
\usepackage[colorlinks,linkcolor=blue]{hyperref}
	\title{Design of Pem}
	\author{PW, Peng Wan}
\begin{document}
	\maketitle
	\tableofcontents
	\section{Overview}
		\subsection{What's Pem?}
			\paragraph{} \textbf{Pem} (\textit{the PW's e-books management}) is an open source toolkit for managing e-books.
		\subsection{What Pem can do?}
			\begin{list}{}{}
				\item Read book file and store to book structure
				\item Write book structure to book file
				\item Modify book structure
			\end{list}
		\subsection{Where I can get Pem?}
			\paragraph{} All Pem implements are hosted on Guthub: \url{https://guthub.com/phylame/pem}
		\subsection{Implements of Pem}
			\subsubsection{Jem} Pem for Java, written with pure Java code.
			\subsubsection{Qem} Pem for Qt, written with Qt frameworks.
		\subsection{License}
			All implements of Pem are under the Apache License 2.0.
	\pagebreak 
	\section{Design}
\end{document}
